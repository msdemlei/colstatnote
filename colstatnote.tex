\documentclass[11pt,a4paper]{ivoa}
\input tthdefs

\title{Towards Blind Discovery 2: Advanced Column Statistics}

\ivoagroup{Registry}

\author[https://wiki.ivoa.net/twiki/bin/view/IVOA/MarkusDemleitner]{Demleitner, M.}

\editor{Demleitner, M.}

% \previousversion[????URL????]{????Concise Document Label????}
\previousversion{This is the first public release}
       

\begin{document}
\begin{abstract}
This note proposes extensions to the Virtual Observatory's registry
ecosystem to make advanced column statistics available for data
discovery.  The guiding use case here is to enable ``blind'' discovery,
i.e., data discovery driven by physical properties of the data rather
than textual descriptions; for instance, people would constrain their
discovery query by ``more or less complete to $20^m$'' rather than
trying a full text search for ``deep survey''.
\end{abstract}


\section*{Conformance-related definitions}

The words ``MUST'', ``SHALL'', ``SHOULD'', ``MAY'', ``RECOMMENDED'', and
``OPTIONAL'' (in upper or lower case) used in this document are to be
interpreted as described in IETF standard RFC2119 \citep{std:RFC2119}.

The \emph{Virtual Observatory (VO)} is a
general term for a collection of federated resources that can be used
to conduct astronomical research, education, and outreach.
The \href{http://www.ivoa.net}{International
Virtual Observatory Alliance (IVOA)} is a global
collaboration of separately funded projects to develop standards and
infrastructure that enable VO applications.


\section{Introduction}

Blind discovery, the location of data relevant to some piece of research
not by prior knowledge of resources or by full text search by using
constraint derived from the data itself (``physical property''), in some
way is the holy grail of data discovery, in particular facilitating the
re-use of specialised observations that may be little known outside of a
specific community but may, for instance, go much deeper than the
well-known global surveys.

In Astronomy, the most important physical constraint for blind discovery
arguably is the the location in space, time, and spectrum.  Making these
properties usable for searches in the VO Registry was the focus of the
Roadmap for Space-Time Discovery in the VO Registry \citep{std:srcReg}.

With a proposed recommendation of VODataService 1.2 \citep{pr:VODS12}
under community review while this note is written, a major part of the
road towards STC-enabled discovery has been taken, and thus it is time
to tackle the next class of constraints useful in discovery: statistical
properties of the data.

To keep the problem tractable, we assume that the data described is
tabular in nature.  This of course does not preclude blind discovery on
collections of array-like data (i.e., observations like images,
spectra, time series).  Such collections are typically published to the
VO through tabular metadata collections, governed by standards like
ObsCore \citep{2017ivoa.spec.0509L} or the upcoming EPN-TAP standard for
publishing solar system data.  Statistics on the columns for such
relational representations can reasonably be expected to be useful for
many types of blind discovery of services serving data sets.  Hence,
this Note only deals with column statistics and leaves the mapping of
array collections to relational representations to other documents.

\subsection{Use Cases}
\label{sect:uc}

Use cases guiding the design of the present proposals include:

\begin{description}
\item[Deep Survey] Give me data for the M32 reaching 25 mag in the
infrared K band.

\item[High Redshift] I am looking for Galaxies with redshifts above 1.

\item[High Precision] I need a catalogue of proper motions with errors
below $0.2\,\rm mas/yr$.

\item[High Precision advanced] I need a catalogue of proper motions with
errors below $0.2\,\rm mas/yr$ at $15^{\rm m}$ in V.

\item[Calibrated] Where are flux-calibrated spectra for stars in
globular clusters?

\item[Planning] A service-spanning query engine wants to estimate how
many columns a query constrained with \verb|WHERE col<30| might yield.
\end{description}


\section{Prior Art}

In VO standards, column statistics have already been part of
VOTable version 1.1 \citep{2004ivoa.spec.0811O}, which defined the
\xmlel{MIN} and \xmlel{MAX} children of \xmlel{VALUES}, as well as
\xmlel{OPTION} elements to declare the values enumerated columns take.
VOTable allows qualifying \xmlel{VALUES} with a \xmlel{type} attribute;
for data discovery, only the case of \verb|actual| is interesting.

When the content model of \xmlel{vs:BaseParam} and its derivatives in
VODataService \citep{2010ivoa.spec.1202P} was designed, VOTable's
\xmlel{FIELD} was obviously taken as a model, but the column statistics
was not taken over.

In consequence, the relational metadata schema for the
\verb|columns| table in TAP \citep{2019ivoa.spec.0927D} even in
version 1.1 does not include any column statistics.

Meanwhile, with the release of Gaia DR1 in 2016, G\'egory Mantelet
experimented with advanced column statistics on his Gaia mirror
\citep{data:arigaia}, exposing them both through the service's TAP
schema -- where extension columns are explictly allowed and even shown
by clients like TOPCAT \citep{soft:topcat} -- and the VOSI tables
endpoint, the latter using VODataService's extension hook of allowing
arbitrary non-namespace attributes on \xmlel{column}.

Mantelet's column statistics includes:

\begin{description}
\item[min\_value, max\_value] As in VOTable with \xmlel{type} \verb|actual|
\item[q1, median, q3} The four quartiles of the distribution of the
values in the column
\item[mean] The column values' mean value
\item[filling] The number of non-NULL values in the column
\end{description}


\section{Proposed Metadata}

Clearly, the nature and amount of metadata to be processed strongly
depends on the use cases envisioned.  For instance, the use case
\emph{Planning} in sect.~\ref{sect:uc} will clearly profit from having
detailed histograms, in particular when one has to deal with multimodal
distributions.  The \emph{High precision advanced} use case would even
need the joint distribution of, in this case, proper motion errors and
magnitudes.

On the other hand, complex metadata statistics are hard to generate,
distribute and use.  It is therefore desirable to identify the
minimal set of metadata that already enables the most important use
cases.

Varying Mantelet's metadata set, we propose the following items for
non-enumerated\footnote{The limit between enumerated and
continuous columns is not a principled one in digital computers; the
practical definition would be: a numeric column is continuous if an
extra value would not have to be explained.} columns:

\begin{description}
\item[min\_value, max\_value] As in Mantelet.  While the items as such are
typically rather unreliable, as for most astrophysically relevant
distributions the tails are sparse, they have obvious interpretations,
allow an estimate of the tailedness of the distribution, and they are
part of the VOTable model.  Hence, it seems prudent to retain them.
\item[percentile03, median, percentile97} the 3rd, 50th, and 97th
percentiles of the distribution of the values in the column.
\item[fill\_factor] The ratio of non-NULL values to the number of rows
in the table.
\end{description}

Changes versus the Mantelet model include:

\begin{itemize}
\item We have dropped the mean, as we believe the median is a robust
alternative that has the additional advantage that it is stable against
common arithmetic operations (e.g., going from parallax to distance).
\item We have replaced the quartiles with the 3rd and 97th percentiles.
This is because we believe that in general having robust estimates for
the extremes of a distribution is more important of data discovery than
understanding its behaviour around its modus.  3 and 97 were used
because for a Gauss distributed random variable, this corresponds to
roughly $2\sigma$ (or 95\% of the values), which seems a good and
intuitive handle for ``characteristic data'').
\item We are dividing Mantelet's filling by the number of rows in the
table in question.  This is to save clients a lookup in a different
table to ascertain this value when filtering out tables that only
occasionally have data of the 

\end{itemize}

\bibliography{ivoatex/ivoabib,ivoatex/docrepo,local.bib}


\end{document}
